\section*{Research \& Development}
\begin{itemize}

  \item \textbf{Predicting Software Developers Technical Skills} (2018 -- Now) \\
  In my PhD project, we are studying methods and techniques that allow us to automatically identify developers skills based on their day-by-day activities. Currently we focus on identifying their expertise in technical skills, e.g., libraries \& frameworks, programming language, etc.

  So far, we published one paper on this subject in a top-ranked conference:
  \begin{itemize}
    \item Identifying Experts on Software Libraries and Frameworks among GitHub Users, in \textit{16th International Conference on Mining Software Repositories (MSR)}, 2019, 1--12 (to appear).
  \end{itemize} 

  \item \textbf{Remote Monitoring Using Wireless Sensor Networks for Dams Oversight} (2016 -- 2017) \\
  In this project  we implemented a low cost solution for monitoring dams using wireless sensor networks. Specifically, I worked by coordinating the development of a real-time monitoring software system, and by implementing a mobile solution for remote data collection.

  As a result, we published two scientific papers in national conferences focused on solutions for electrical companies, and submitted 5 patent applications.

  \item \textbf{Documenting Application Programming Interfaces with Source Code Examples} (2011 -- 2013) \\
  In my MSc project, we implemented a new technique that enriches APIs JavaDocs with source code examples extracted from real systems. For this, we developed a simple heuristic based on program slicing that extracts intra-method source code snippets from Java source code. An Android version of the platform is available at \url{http://apiminer.org/}.

  The findings achieved in this project are reported in three conferences publications:
  \begin{itemize}
    \item Documenting APIs with Examples: Lessons Learned with the APIMiner Platform, in \textit{20th Working Conference on Reverse Engineering (WCRE)}, 2013, 1--8.
    \item APIMiner 2.0: Uma Plataforma para Recomendação de Exemplos de Uso de APIs Baseados em Padrões de Uso, \textit{IV Congresso Brasileiro de Software: Teoria e Prática (Sessão de Ferramentas)}, 2013. 
    \item APIMiner: Uma Plataforma para Recomendação de Exemplos de Uso de APIs \textit{III Congresso Brasileiro de Software: Teoria e Prática (Sessão de ferramentas)}, 2012.
  \end{itemize}  

  \item \textbf{Investigating Warnings Reported by Bug Finding Tools} (2008 -- 2010) \\
  In this project we investigated the effectivness of warnings reported by static analysis tools, such as FindBugs, Checkstlye, etc. More specifically, we studied how these tools can contribute to maintain software quality and whether the warnings reported by them really help in catching new bugs.
  
  Our findings are reported in the following publications:
  \begin{itemize}
    \item Static Correspondence and Correlation between Field Defects and Warnings Reported by a Bug Finding Tool, 2012, {\it Software Quality Journal}, Springer, 241--257.
    \item Study on the Relevance of the Warnings Reported by Java Bug-Finding Tools, 2011, {\it IET Software}, IET, 366--374.
    \item Avaliação da Relevância dos Warnings Reportados por Ferramentas de Análise Estática. \textit{XXX Concurso de Trabalhos de Iniciação Científica (CTIC)}, 2011.
    \item Um Estudo sobre a Correlação entre Defeitos de Campo e Warnings Reportados por uma Ferramenta de Análise Estática \textit{IX Simpósio Brasileiro de Qualidade de Software}, 2010.
    \item Os Defeitos Detectados pela Ferraamenta de Análise Estática FindBugs são Relevantes? \textit{IX Simpósio Brasileiro de Qualidade de Software}, 2010.
    \item Uma Meta-Ferramenta para Detecção de Defeitos. \textit{VII Workshop de Manutenção de Software Moderna}, 2010. 
  \end{itemize}
\end{itemize}